\documentclass[a4paper, 12pt]{report}

% base
\usepackage[utf8]{inputenc}

% packages
\usepackage{upquote}
\usepackage{tikz}
\usepackage{amsmath}
\setcounter{MaxMatrixCols}{11}
\usepackage{listings}
\usepackage{xcolor}
\usepackage{algorithm}
\usepackage{algpseudocode}
\usepackage{enumitem}

% references
\usepackage{hyperref}
\hypersetup{
    colorlinks=true,
    linkcolor=black,
    filecolor=black,
    urlcolor=black,
    citecolor=black
}

% code
\definecolor{codegreen}{rgb}{0,0.6,0}
\definecolor{codegray}{rgb}{0.5,0.5,0.5}
\definecolor{codepurple}{rgb}{0.58,0,0.82}
\definecolor{backcolour}{rgb}{0.95,0.95,0.92}
\lstdefinestyle{mystyle}{
    backgroundcolor=\color{backcolour},   
    commentstyle=\color{codegreen},
    keywordstyle=\color{magenta},
    numberstyle=\tiny\color{codegray},
    stringstyle=\color{codepurple},
    basicstyle=\ttfamily\footnotesize,
    breakatwhitespace=false,         
    breaklines=true,                 
    captionpos=b,                    
    keepspaces=true,                 
    numbers=left,                    
    numbersep=5pt,                  
    showspaces=false,                
    showstringspaces=false,
    showtabs=false,                  
    tabsize=2
}
\lstset{style=mystyle}


% margine
\usepackage[margin=1.5in]{geometry} 

% paragraphs
\usepackage{parskip}
\setlength{\parindent}{0pt}

% impaginazione
\usepackage{fancyhdr}
\pagestyle{fancy}
\fancyhf{}
\rhead{\thepage}
\renewcommand{\headrulewidth}{0pt}

% titles
\usepackage{titlesec}
\titleformat{\chapter}[block]{\Huge\bfseries}{\thechapter.}{1em}{}
\titlespacing*{\chapter}{0pt}{-20pt}{10pt}

% frontpage va messa prima del begin nei report

% for references in table of contents, last package to upload!
\usepackage{hyperref}

\begin{document}

%---

\begin{titlepage}
\centering
\includegraphics[width=0.35\textwidth]{logo.png}\par\vspace{1cm}
\vspace{1.25cm}
{\huge\bfseries A Node Removal Algorithm for Food Webs \par}
\vspace{0.5cm}
{\Large\bfseries Leveraging Edge Directionality for \par Improved Simulation Outcomes \par}
\vspace{4cm}
{\Large\itshape Alain Joss \par}
\vfill
{\large \today \par}
\end{titlepage}

%---

\tableofcontents
\thispagestyle{empty}
\newpage


\chapter{Assessing Robustness}

\section{The Overall Idea}
For assessing the robustness of a network, 
the idea is to perform some kind of perturbation, 
and compute different kinds of measures in the process,
until a certain state has been reached.
The high level algorithm looks as follows:

\begin{algorithm}
    \caption{Perturbation}
    \begin{algorithmic}[1]
    \small
    \Require Graph G
    \State Initialize $stop\_state \gets false$
    \State Initialize $robustness\_states \gets list()$
    \While{not $stop\_state$}
        \State Remove a node from G (according to a predefined attack strategy).
        \State Compute some metric of robustness of G.
        \State Save value of metric in $robustness\_states$
        \If{some network state} 
            \State $stop\_state \gets true$
        \EndIf
    \EndWhile
    \State Visualize the robustness trend of the perturbation.
    \end{algorithmic}
\end{algorithm}

By visualizing the trend of robustness across diverse perturbations, we can evaluate the network's resilience under a variety of metrics and attack strategies. 
In this context, an "attack strategy" refers to the specific probability distribution employed for node removal. 

\section{Non-Deterministic Perturbations}
If at any step, a perturbation involves a non-deterministic computation,
its results can be skewed by the effect of randomness.
The solution is to run multiple times the same perturbation, 
and average the results across runs:

\begin{minipage}{\textwidth}
\begin{algorithm}[H]
    \caption{Multiple Perturbations}
    \begin{algorithmic}[1]
    \small
    \Require Graph $G$, Number of perturbations $N$
    \For{$i = 1$ to $N$}
        \State Perform perturbation $i$ on $G$.
    \EndFor
    \State Average the perturbations.
    \State Visualize the robustness trend of the perturbations.
    \end{algorithmic}
\end{algorithm}
\end{minipage}

\section{Node Removal in Food Webs}
\subsection{Types of Extinctions}
Food web perturbations can result in two types of node removals: primary and secondary extinctions. Primary extinctions refer to a node's direct removal based on a defined attack strategy. Secondary extinctions occur when a node's elimination, directly or indirectly, results in the removal of another node. Regardless of the attack strategy employed, primary extinctions occur one by one.

\subsection{The Undirected Approach}
Considering the undirected version of a food web, calculating secondary extinctions involves assessing if a node becomes isolated after each primary extinction. Isolation can result from either primary or secondary extinctions. The algorithm for the undirected approach is as follows:

\begin{algorithm}
    \caption{Undirected Removal}
    \begin{algorithmic}[1]
    \small
    \Require Graph $G$, Node $n$
    \State Remove node $n$ from $G$ 
    \State Initialize $continue\_flag$ $\gets$ true
    \While{$continue$}
        \State $continue\_flag$ $\gets$ false
        \For{each Node $u$ in $G$}
            \If{$u$.degree == 0}
                \State $continue\_flag$ $\gets$ true
                \State Remove node $u$ from $G$
            \EndIf
        \EndFor
    \EndWhile
    \end{algorithmic}
\end{algorithm}

The $continue\_flag$ ensures we only check for isolated nodes as long as necessary, as the removal of one node can potentially leave others isolated.


\subsection{The Directed Approach}
Given that food webs are typically directed networks, a more appropriate node-removal algorithm would consider the direction of prey-predator relationships. This means when a node is removed, all species feeding on it also go extinct if they have no other food sources. Calculating secondary extinctions here involves checking if a node still receives energy (has incoming edges). The algorithm for this directed approach is as follows:

\begin{minipage}{\textwidth}
\begin{algorithm}[H]
    \caption{Directed Node Removal}
    \begin{algorithmic}[1]
    \small
    \Require Directed Graph $G$, Node $n$
    \State $k\_level\_neighbors \gets$ successors($n$) in $G$
    \State Remove node $n$ from $G$
    \While{$k\_level\_neighbors \neq \emptyset$}
        \State Initialize empty sets $new\_level\_neighbors$, $removed\_neighbors$
        \For{each Node $neighbor$ in $k\_level\_neighbors$}
            \State $change\_flag \gets \text{false}$
            \If{$G$.in\_degree($neighbor$) == 0 or \\ 
            \hspace{\algorithmicindent} \hspace{\algorithmicindent} $G$.degree($neighbor$) == 0 or \\
            \hspace{\algorithmicindent} \hspace{\algorithmicindent} $G$.in\_degree($neighbor$) == 1 and \\ 
            \hspace{\algorithmicindent} \hspace{\algorithmicindent} $G$.has\_edge($neighbor$, $neighbor$)}
            \State Add successors($neighbor$) to $new\_level\_neighbors$
            \State Add $neighbor$ to $removed\_neighbors$
            \State Remove $neighbor$ from $G$
            \State $change\_flag \gets \text{true}$
            \EndIf
            \If{not $change\_flag$}
                \State Break
            \EndIf
        \EndFor
        \State $k\_level\_neighbors \gets new\_level\_neighbors \setminus removed\_neighbors$
    \EndWhile
    \end{algorithmic}
\end{algorithm}
\end{minipage}

Node removal occurs when:
\begin{itemize}
\item "G.in degree(neighbor) == 0": A node has no food source.
\item "G.degree(neighbor) == 0": A node becomes completely isolated, including plants, which we assume need pollinators.
\item "G.in degree(neighbor) == 1 and G.has edge(neighbor, neighbor)": A cannibalistic node is not receiving energy from the network.
\end{itemize}

\section{Attack Strategies}
Attack strategies dictate the rules governing the selection of nodes (species) for primary extinction. These nodes are then processed by the node removal algorithm. The implemented strategies include:

\begin{itemize}
\item Random Attack: Applies an equal probability for removal to all nodes.
\item Centrality Attack: Prioritizes node removal based on a chosen centrality measure, proceeding in descending order. This strategy produces deterministic results.
\item Habitat Attack: Nodes associated with a specific habitat have an increased removal probability. The strategy also considers the number of different habitats a node inhabits.
\item Threatened Species Attack: Nodes representing species on a threatened species list are prioritized for removal. Within this list, the removal order is random.
\end{itemize}

\section{Summary}

Putting it all together, the robustness assessment of the food web looks as follows:

\begin{minipage}{\textwidth}
\begin{algorithm}[H]
    \caption{Robustness Assessment}
    \begin{algorithmic}[1]
    \small
    \Require Graph $G$, Number of perturbations $N$, Attack Strategy $S$
    \State Initialize $all\_robustness\_states \gets list()$
    \For{$i = 1$ to $N$}
    \State Initialize $stop\_state \gets false$
    \State Initialize $robustness\_states \gets list()$
    \While{not $stop\_state$}
    \State Choose a node according to attack strategy $S$ and remove from $G$.
    \State Apply "Directed Node Removal" algorithm.
    \State Compute some metric of robustness of $G$.
    \State Save value of metric in $robustness\_states$
    \If{some network state}
    \State $stop\_state \gets true$
    \EndIf
    \EndWhile
    \State Add $robustness\_states$ to $all\_robustness\_states$
    \EndFor
    \State Compute the average robustness trend from $all\_robustness\_states$.
    \State Visualize the averaged robustness trend of the perturbations.
    \end{algorithmic}
\end{algorithm}
\end{minipage}
\chapter{Addressing Key Concerns}

\section{Preliminary Steps}
In preparation for the network's perturbation analysis, we need to address two concerns as follows:
\begin{enumerate}
\item For nodes with an in-degree of zero, that are not basal nodes, we construct randomized inward-links.
\item As the network under consideration is a meta web, rather than a real food web, we eliminate k\% of inward links of generalist species.
\end{enumerate}

\section{Correcting Fake Basal Species}
This step rectifies species misrepresented as basal due to data gaps. We create randomized inward-links for these species, considering their diet, habitat, and zone attributes.

\begin{algorithm}
    \caption{Randomized Inward-Links Construction}
    \begin{algorithmic}[1]
    \small
    \Require Dataset "species\_for\_randomized\_link\_assignment", Dataset "all\_species\_and\_feeding\_groups"
    \For{each row in "species\_for\_randomized\_link\_assignment"}
        \State Extract the species, its diet, diet rank, habitat, and zone.
        \State Check the second dataset "all\_species\_and\_feeding\_groups" to find species that share the same habitat and diet rank. 
        \State Link the fake basal species to these species.
        \State Remove duplicate links.
        \State Retain a subset of potential links.
        \State Add these links to the overall network graph.
    \EndFor
    \end{algorithmic}
\end{algorithm}
Depending on the species' dietary breadth, we retain different numbers of potential links:

\begin{itemize}
\item For Generalized-diet species, we retain 5\% of potential interaction links.
\item For Specialized-diet species, we randomly select between 1 and 5 potential interaction links.
\end{itemize}

\section{Managing Generalist Species}

Here, we eliminate p\% of inward links for generalist species, guided by the Optimal Foraging Theory (OFT). Generalist species are defined as those with a degree threshold of k. The algorithm is as follows:


\begin{algorithm}
    \caption{Prune Network on Degree Threshold}
    \begin{algorithmic}[1]
    \small
    \Require Network $G$, Degree threshold $k$, Percentage to remove $p$
    \State $nodes \gets$ List of nodes in $G$
    \For{each $node$ in $nodes$}
    \State $inDegree \gets$ in-degree of $node$ in $G$
    \If{$inDegree \geq k$} \Comment{The node is a generalist}
    \State $inwardEdges \gets$ List of in-edges of $node$ in $G$
    \State $numEdgesToRemove \gets inDegree \times p$
    \State $edgesToRemove \gets$ randomly sample $numEdgesToRemove$ from $inwardEdges$
    \State Remove $edgesToRemove$ from $G$
    \EndIf
    \EndFor
    \Ensure Pruned network $G$
    \end{algorithmic}
    \end{algorithm}
    In this context, a node's in-degree refers to the number of incoming edges (number of predators). If a node's in-degree equals or exceeds a threshold k, it's deemed a generalist. For each generalist, a proportion p of its inward edges is randomly selected for removal, effectively pruning the network.

\section{Summary}

To summarize, the following steps prepare the network for its robustness assessment:

\begin{algorithm}
    \caption{Network Preparation}
    \begin{algorithmic}[1]
    \small
    \Require Graph $G$, Datasets $D1$, $D2$
    \State Apply "Randomized Inward-Links Construction" using $D1$ and $D2$ to $G$.
    \State Apply "Pruning Based on OFT" to $G$.
    \end{algorithmic}
\end{algorithm}


\chapter{The Final Pipeline}

\begin{itemize}
    \item importing the dataset
    \item creating new links
    \item removing links
    \item create attack strategies $S$
    \item perform perturbations for $|S|$ strategies
    \item visualize all the different perturbation results
\end{itemize}


\end{document}





